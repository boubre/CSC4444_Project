\documentclass[mathserif]{beamer}
\usepackage{mathtools}
\usepackage{amssymb}
\usetheme{Copenhagen}

\begin{document}
	\title[Analyzing Stochastic Time-Series Data]{Techniques for Analyzing Stochastic Time-Series Data}
	\author[Castleberry \and Oubre \and Yu]{Dennis Castleberry \and Brandon Oubre \and Haikou Yu}
	\date{\today}
	\frame{\titlepage}
	
	\section{The Naive Bayes Classifier}
	\subsection{Overview}
	\begin{frame}
		\frametitle{The Naive Bayes Classifier}
		\begin{itemize}
			\item Reduce classification to probability. What is \(P(class | attribute1, attribute2, ..., attributeN)\).
			\item Assumes that each attribute is independent of the others. (Hence the ``Naive'' nickname.)
			\item For example, let's consider if a car is stolen using \(P(stolen | Color, Type)\). Naive Bayes will assume \(color=red\) and \(type=sportscar\) to be independent.
			\item Naive Bayes is not sensitive to irrelevant attributes, since the probabilities of such attributes will be similar for all classes.
		\end{itemize}
	\end{frame}
	
	\begin{frame}
		\frametitle{Advantages and Disadvantages of Naive Bayes}
		\textbf{Advantages}
		\begin{itemize}
			\item Only requires a single scan to train.
			\item Fast classification.
			\item Handles real and discrete data.
			\item Not sensitive to irrelevant attributes.
		\end{itemize}
		\textbf{Disadvantages}
		\begin{itemize}
			\item Assumes all attributes to be independent.
		\end{itemize}
	\end{frame}
	
	\subsection{Classification}
	\begin{frame}
		\frametitle{Training the Classifier}
	\end{frame}
		
\end{document}